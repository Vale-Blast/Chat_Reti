\section{Conclusioni e sviluppi futuri}
Il lavoro svolto ha permesso un completo sviluppo di un'applicazione di
messaggistica pienamente funzionante e operativa su reti locali. I test
fatti in locale e su VPN hanno dato risultati soddisfacenti permettendo un
pieno scambio di messaggi e file (si è testato, ad esempio, l'invio di
una semplice immagine e di qualche file di testo). L'uso dell'interfaccia grafica realizzata con
SWING - siamo certi - permette una visualizzazione rapida dei messaggi
e semplifica molto l'uso dell'applicazione anche ad
utenti meno avvezzi all'uso di un terminale.
Inoltre durante le ultime prove abbiamo verificato ulteriormente la correttezza del
nostro lavoro provando a catturare i pacchetti con \emph{Wireshark} (\url{https://www.wireshark.org/})
e abbiamo osservato che i messaggi in chiaro erano leggibili mentre gli altri messaggi erano effettivamente
criptati e quindi illeggibili a chiunque faccia sniffing della rete.

Nonostante l'applicazione abbia raggiunto una certa solidità,
numerosi sono i possibili interventi da attuare per migliorarla.
\begin{itemize}
	\item Sarebbe interessante criptare anche gli allegati oltre che i
	messaggi al fine di garantire una maggiore riservatezza;
	\item I file contenenti le chat sono memorizzati sul PC
	dell'utente in chiaro, il che li rende vulnerabili.
	Di fatto, si potrebbero cifrare e decifrare ogni volta che 	
	viene chiamato il metodo \texttt{openChat} (ossia ogni volta che viene premuto un pulsante nella barra a 
	sinistra dell'applicazione), introducendo tuttavia in questo modo un overhead che,
	durante lo sviluppo, abbiamo
	deciso di non aggiungere per non appesantire ulteriormente il programma;
	\item Un discorso molto simile vale per il file di configurazione che pur 
	essendo nascosto potrebbe essere	modificato;
	\item Si potrebbe aggiungere un supporto Android per l'applicazione, utilizzandola
	per far comunicare più utenti nella stessa rete Wi-fi;
	\item La sicurezza della chat è migliorabile se si aggiunge anche un controllo
	per mezzo di un certificato oppure un supporto per l'inserimento di una firma
	digitale durante lo scambio di messaggi;
	\item Un altro problema presente che potrebbe essere risolto in futuro è legato ai nickname:
	la cronologia dei messaggi viene salvata, come detto precedentemente, in un file 
	\texttt{nickname.txt};
	Questa decisione sta in piedi fintanto che non esistono due utenti con lo stesso nickname
	(visto che ipotizziamo che un utente non  possa cambiare nickname se non modificando il file
	\emph{.chat}), nel qual caso una sovrapposizione è 
	inevitabile, portando alla perdita di entrambi gli storici delle conversazioni
	che si mischierebbero in un unico file.
	Una soluzione ipotizzata comporta l'inserimento di un server centrale predisposto alla gestione dei 
	nickname;
	\item Altri miglioramenti attuabili riguardano poi prettamente l'interfaccia grafica
	del programma, inserendo la 
	possibilità di cambiare colori, trasformando i messaggi in nuvolette, cambiare
	lo sfondo della conversazione e 
	altre modifiche per rendere l'applicazione più user-friendly;
	\item Potrebbe essere utile dare la possibilità all'utente di visualizzare le informazioni relative a ogni 
	messaggio come il timestamp di invio e ricezione, la lettura di tale messaggio da parte dell'utente;
	\item Una modifica utile sarebbe l'introduzione dei gruppi, 
	come in qualsiasi applicazione di messaggistica.
\end{itemize}