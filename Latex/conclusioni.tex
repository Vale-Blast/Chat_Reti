\section{Conclusioni e sviluppi futuri}
Il lavoro svolto ha permesso un completo sviluppo di un'applicazione di
messaggistica pienamente funzionante e operativa su reti locali. I test
fatti in locale e su VPN hanno dato risultati soddisfacenti permettendo un
pieno scambio di messaggi e file (si è testato, ad esempio, l'invio di
una semplice immagine). L'uso dell'interfaccia grafica realizzata con
SWING - siamo certi - permette una visualizzazione rapida dei messaggi
e semplifica molto l'uso dell'applicazione anche ad
utenti meno avvezzi all'uso di un terminale.

Nonostante l'applicazione abbia raggiunto una certa solidità,
numerosi sono i possibili interventi da attuare per migliorarla.
Sarebbe interessante criptare anche gli allegati oltre che i
messaggi al fine di garantire una maggiore riservatezza.
Inoltre i file contenenti le chat sono memorizzati sul PC
dell'utente in chiaro, il che li rende vulnerabili.
Di fatto, si potrebbero cifrare e decifrare ogni volta che viene chiamato
il metodo \texttt{openChat} (ossia ogni volta che viene premuto
un pulsante nella barra a sinistra dell'applicazione), introducendo
tuttavia in questo modo un overhead che, durante lo sviluppo, abbiamo
deciso di non aggiungere per non appesantire ulteriormente il programma.
Un altro problema presente che potrebbe essere risolto in futuro è il seguente:
la cronologia dei messaggi viene salvata, come detto precedentemente, in un
file
$$ \texttt{nickname.txt} $$
Questa decisione sta in piedi fintanto che non esistono due utenti con lo stesso nickname,
nel qual caso una sovrapposizione è inevitabile, portando alla perdita definitiva
di uno dei due storici. Una soluzione ipotizzata comporta l'inserimento di
un server centrale predisposto alla gestione dei nickname.
Altri miglioramenti attuabili riguardano poi prettamente l'interfaccia
grafica del programma, inserendo la possibilità di cambiare colori,
trasformando i messaggi in nuvolette, e così via.