\section{Il package net}
Nel package \texttt{net} sono organizzate tutte le classi che si occupano di effettuare e gestire lo
scambio dei messaggi e di ricercare altri utenti nella rete.

\subsection{Ricerca di utenti}
La classe che si occupa della scansione è la classe \texttt{scan}. Per effettuare la
ricerca viene utilizzato \texttt{Nmap},  un software libero distribuito con licenza GNU GPL
creato originariamente per effettuare port scanning, cioè mirato all'individuazione di porte aperte
su un computer bersaglio o anche su range di indirizzi IP, in modo da determinare quali
servizi di rete siano disponibili. Un tipico esempio dell'uso di \texttt{Nmap} è il seguente:

\begin{lstlisting}
nmap www.google.it

Starting Nmap 6.40 ( http://nmap.org ) at 2015-06-05 09:47 CEST
Nmap scan report for www.google.it (74.125.206.94)
Host is up (0.026s latency).
Not shown: 998 filtered ports
PORT    STATE SERVICE
80/tcp  open  http
443/tcp open  https
\end{lstlisting}

Nello specifico, poi, per quanto concerne la nostra applicazione, \texttt{Nmap}
viene utilizzato principalmente per individuare chi utilizza il servizio di messaggistica.
Per eseguire \texttt{Nmap} viene utilizzata la classe \texttt{Process} di Java.
Inoltre, quando viene avviata una nuova scansione, tutte le coppie
$$ \langle \text{Nickname,indirizzo} \rangle $$
trovate fin'ora vengono buttate via per essere sostituite da quelle identificate per mezzo
della nuova scansione.

\subsection{Scambio di messaggi}
Lo scambio di messaggi non sfrutta, come si è intuito dai paragrafi precedenti un'architettura
client-server, ma il peer-to-peer. Fondamentalmente, tutti i nodi sono equivalenti e possiedono
un server sempre in ascolto per la ricezione dei messaggi ed un client predisposto, invece, all'invio
dei messaggi stessi.