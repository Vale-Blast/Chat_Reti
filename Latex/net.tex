\section{Il package net}
Nel package \texttt{net} sono organizzate tutte le classi che si occupano di effettuare e gestire lo
scambio dei messaggi e di ricercare altri utenti nella rete.

\subsection{Ricerca di utenti}
La classe che si occupa della scansione è la classe \texttt{Scan}. Per effettuare la
ricerca viene utilizzato \texttt{Nmap} (\url{http://nmap.org/}),  un software libero distribuito
con licenza GNU GPL creato originariamente per effettuare port scanning, cioè mirato
all'individuazione di porte aperte su un computer bersaglio o anche su range di indirizzi IP,
in modo da determinare quali servizi di rete siano disponibili. Un tipico esempio dell'uso
di \texttt{Nmap} è il seguente:

\begin{lstlisting}
nmap www.google.it

Starting Nmap 6.40 ( http://nmap.org ) at 2015-06-05 09:47 CEST
Nmap scan report for www.google.it (74.125.206.94)
Host is up (0.026s latency).
Not shown: 998 filtered ports
PORT    STATE SERVICE
80/tcp  open  http
443/tcp open  https
\end{lstlisting}
\leavevmode
\\
Nello specifico, poi, per quanto concerne la nostra applicazione, \texttt{Nmap}
viene utilizzato principalmente per individuare quali host sono presenti nella LAN.
Per eseguire \texttt{Nmap} viene adoperata la classe \texttt{Process} di Java.
Quando viene avviata una nuova scansione si invia a tutti gli host raggiungibili in LAN
un messaggio del tipo \emph{\#\#NICK:$nickname$\#\#} in cui ``nickname'' rappresenta il nickname
dell'utente. Chi lo riceve risponderà con il proprio nickname. Tutte le coppie degli 
host raggiungibili che hanno risposto vengono salvate tra gli host conosciuti in una mappa
a doppia chiave come coppia 
$$ \langle \text{Nickname, Indirizzo} \rangle $$
e vengono mostrati nella barra laterale sinistra della GUI, come già anticipato sopra.
Osserviamo in particolare che la classe \texttt{Scan}
estende la classe \texttt{Thread} costituendo quindi un thread a parte rispetto al resto
dell'applicazione.

\subsection{Scambio di messaggi}
Lo scambio di messaggi non sfrutta, come si è intuito dai paragrafi precedenti un'architettura
client-server, ma il peer-to-peer. Fondamentalmente, tutti i nodi sono equivalenti e possiedono
un thread server sempre in ascolto per la ricezione dei messaggi ed un client predisposto, invece, all'invio
dei messaggi stessi. Abbiamo creato dunque una classe \texttt{Message} che implementa l'ADT (Abstract Data Type)
\textit{messaggio}, contenente fondamentalmente un campo di tipo stringa per il testo, un campo
per la data e un capo \texttt{tipo} posto a $0$ se il messaggio è stato ricevuto e posto a $1$ se inviato.

Lo scambio vero e proprio è implementato invece nelle classi \texttt{Chat\char`_manager} e \texttt{Server},
cuore dell'intero programma. Nella prima vengono svolti vari compiti:
\begin{itemize}
	\item caricamento, e eventualmente aggiornamento, del file audio che verrà poi riprodotto per notificare 
	l'arrivo di un messaggio o di un allegato;
	\item avvio dei thread relativi alla classe \texttt{Server} e alla classe \texttt{Scan};
	\item creazione, aggiornamento e gestine della struttura dati \emph{HashBiMap} usata per memorizzare gli host
	conosciuti;
	\item invio dei messaggi e degli allegati (alla pressione dei pulsanti mostrati nella Figura \ref{fig:gui})
	tramite i metodi \texttt{send()} e \texttt{attach()};
	\item la gestione di una mappa a doppia chiave, realizzata anch'essa tramite la classe \texttt{HashBiMap},
	per contenere i messaggi inviati che si è sicuri dell'arrivo. A ogni invio la coppia
	$$ \langle \text{Destinatario, Messaggio} \rangle$$
	è aggiunta alla mappa e verrà tolta dalla struttura dati solo alla ricezione di un ack da quel destinatario
	per quel messaggio. L'invio di qualsiasi tipo di messaggio (messaggio normale, allegato, messaggio ``speciale'')
	è effettuato tramite la classe \texttt{DatagramSocket} via UDP.
\end{itemize}
I messaggi speciali sono dei messaggi usati per la sincronizzazione tra gli host e la gestione della rete:
\begin{itemize}
	\item[-] \emph{\#\#NICK:$nickname$\#\#} Come visto prima questo messaggio viene inviato dalla classe
	\texttt{Scan} a tutti gli host raggiungibili in modo che conoscano il nickname dell'utente;
	\item[-] \emph{\#\#NICK:$nickname$\#\$\#} Questo messaggio viene inviato in automatico dalla classe 
	\texttt{Server} come risposta alla ricezione di un messaggio \emph{\#\#NICK:$nickname$\#\#}, è codificato
	in maniera diversa dal 	precedente in modo che il Server della controparte non risponda a tale messaggio;
	\item[-] \emph{\#\#RESZ:$size$\#\#} Questo messaggio viene inviato appena prima dell'invio di un allegato in 
	modo che il Server del destinatario allarghi in automatico il buffer di ricezione abbastanza da ricevere 
	l'allegato. Tale buffer verrà poi in automatico reimpostato al default di 256 Byte;
	\item[-] \emph{\#\#RCVD:$message$\#\#} È l'ACK nominato sopra che il Server invia in automatico alla ricezione 	
	di un messaggio (né allegato né messaggio speciale) al mittente in modo da fargli sapere che il messaggio è 
	stato ricevuto;
	\item[-] \emph{\#\#DOWN\#\#} Come detto precedentemente è il messaggio che l'applicazione manda a tutti gli 
	host conosciuti appena prima di essere chiusa.
\end{itemize}

La classe \texttt{Server} ha come scopo principale quello di avviare il server
inizializzando poi il campo \texttt{myIP} e permettendo dunque ad altri utenti di poterlo identificare.
Dopodiché si mette in attesa di eventuali datagrammi. Alla ricezione di ogni datagramma controlla se
è criptato (messaggi) o in chiaro (allegati e messaggi speciali) e se necessario lo decripta. 
I messaggi speciali, come visto precedentemente, vengono gestiti in automatico dal Server, per i
messaggi e per gli allegati viene invece interpellata la classe \texttt{Chat\char`_manager} che riprodurrà
il suono di notifica e salverà nel file \emph{$nickname$.txt} il messaggio o il nome dell'allegato. In caso di 
allegato, aprirà automaticamente una finestra per chiedere all'utente dove salvare l'allegato ricevuto.
Se invece si tratta di un messaggio e la chat con l'host mittente è già aperta viene aggiornata la schermata e 
aggiunto il messaggio alla chat visualizzata. Nel caso non sia attualmente aperta la chat con l'host
mittente oltre alla riproduzione del suono di notifica viene colorato di rosso il bottone laterale.