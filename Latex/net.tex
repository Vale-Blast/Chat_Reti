\section{Il package net}
Nel package \texttt{net} sono organizzate tutte le classi che si occupano di effettuare e gestire lo
scambio dei messaggi e di ricercare altri utenti nella rete.

\subsection{Ricerca di utenti}
La classe che si occupa della scansione è la classe \texttt{scan}. Per effettuare la
ricerca viene utilizzato \texttt{Nmap} (\url{http://nmap.org/}),  un software libero distribuito con licenza GNU GPL
creato originariamente per effettuare port scanning, cioè mirato all'individuazione di porte aperte
su un computer bersaglio o anche su range di indirizzi IP, in modo da determinare quali
servizi di rete siano disponibili. Un tipico esempio dell'uso di \texttt{Nmap} è il seguente:

\begin{lstlisting}
nmap www.google.it

Starting Nmap 6.40 ( http://nmap.org ) at 2015-06-05 09:47 CEST
Nmap scan report for www.google.it (74.125.206.94)
Host is up (0.026s latency).
Not shown: 998 filtered ports
PORT    STATE SERVICE
80/tcp  open  http
443/tcp open  https
\end{lstlisting}

Nello specifico, poi, per quanto concerne la nostra applicazione, \texttt{Nmap}
viene utilizzato principalmente per individuare chi utilizza il servizio di messaggistica.
Per eseguire \texttt{Nmap} viene adoperata la classe \texttt{Process} di Java.
Quando viene avviata una nuova scansione, tutte le coppie
$$ \langle \text{Nickname,indirizzo} \rangle $$
trovate vengono salvate e mostrate nella barra laterale sinistra della GUI, come già
anticipato sopra.

\subsection{Scambio di messaggi}
Lo scambio di messaggi non sfrutta, come si è intuito dai paragrafi precedenti un'architettura
client-server, ma il peer-to-peer. Fondamentalmente, tutti i nodi sono equivalenti e possiedono
un server sempre in ascolto per la ricezione dei messaggi ed un client predisposto, invece, all'invio
dei messaggi stessi. Abbiamo creato dunque una classe \texttt{Message} che implementa l'ADT
\textit{messaggio}, contenente fondamentalmente un campo di tipo stringa per il testo, un campo
per la data e un capo \texttt{tipo} posto a $0$ se il messaggio è ricevuto e posto a $1$ se inviato.

Lo scambio vero e proprio è implementato invece nelle classi \texttt{Chat\char`_manager} e \texttt{Server},
cuore dell'intero programma. Nella prima vengono settati i suoni, avviato il server, avviata la scansione
con \texttt{Nmap} della rete locale, aggiunti eventuali altri utenti che usano la chat, etc. Inoltre
è presente l'indispensabile metodo \texttt{send} che invia messaggi con Datagram Socket. Ad ogni
invio segue un acknowlegement di conferma di avvenuta ricezione.
Il metodo \texttt{attach} invece consente la gestione corretta degli allegati: un file viene, di
fatto, convertito in un flusso di byte, inglobato in Datagram Socket e inviato senza essere stato cifrato.

La classe \texttt{Server}, d'altro canto, ha come scopo principale quello di avviare il server
inizializzando poi un campo myIP e permettendo dunque ad altri utenti di poterlo identificare.
Dopodiché si addormenta, mettendosi in attesa di messaggi.