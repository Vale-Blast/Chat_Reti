\section{Il package app}
Il codice è stato diviso in due package: uno che si occupa di gestire l'applicazione
e la codifica dei messaggi, e l'altro che invece contiene tutte le classi inerenti
allo scambio e alla trasmissione dei messaggi stessi.
Il package \texttt{app} contiene tutte e sole le classi che si occupano del corretto
funzionamento dell'applicazione, a partire dalla definizione della schermata di login
fino ad arrivare alla gestione delle chiavi. Analizzeremo più dettagliatamente
nei prossimi sotto-paragrafi l'uso delle singole classi.

\subsection{Avvio dell'applicazione}
La classe \texttt{Main} è la classe principale dell'applicazione.
Possiede un unico metodo \texttt{main} che
si occupa di lanciare il programma eseguendo \texttt{run},
ammesso che l'utente abbia effettuato il login in questa
o in una precedente sessione.
In caso contrario, viene fatto scegliere all'utente un nickname da usare nella chat
mediante il pannello grafico mostrato in precedenza e,solo dopo questa operazione,
l'applicazione può essere correttamente utilizzata.

\subsection{Interfaccia Grafica}
La classe \texttt{App}, che implementa \texttt{Runnable}, si occupa della
creazione e della gestione della finestra grafica della chat. La GUI,
in particolare, è stata realizzata mediante Swing, un framework per Java
orientato allo sviluppo di interfacce grafiche.
I principali metodi qui presenti si occupano, quindi, di generare la GUI, ma non
solo: le conversazioni avvenute fra gli utenti vengono salvate in locale e, se
la conversazione con un utente già noto viene ripresa in un qualche futuro, allora
i precedenti messaggi scambiati con lui sono caricati da un file del tipo
$$ \texttt{nickname.txt} $$ 
e mostrati nella schermata principale dell'applicazione.

Il file di configurazione \texttt{.chat}, invece, è un file che permette di
tenere traccia del nickname dell'utente. Quando avvia l'applicazione per la
prima volta, infatti, compare la schermata grafica mostrata in precedenza per l'inserimento
di un username. Il tasto \texttt{OK} è disabilitato fintanto che l'utente non ha
scritto almeno una stringa valida, ossia una stringa più lunga di zero caratteri.
Premendo il tasto di conferma, viene salvata nel file con estensione \texttt{.chat}
una stringa del tipo:
$$ \texttt{NICK: nickname} $$
insieme ad altre informazioni di configurazione, quali time to sleep, buffer lenght, etc.

\subsection{Cifratura dei messaggi}
La classe principale che implementa la cifratura dei messaggi è la classe \texttt{Encryption}.
La codifica sfrutta l'algoritmo a chiave pubblica RSA (dal nome di coloro che
lo hanno proposto, Rivest, Shamir, Adleman), il quale utilizza operazioni in modulo per generare
le chiavi e l'assunzione che violare la chiave privata sia un problema computazionalmente non
trattabile. Per realizzarlo, è stata usata la classe \texttt{KeyPair} di Java, che permette di
generare una coppia di chiavi (pubblica e privata). Queste vengono estrette e salvate in locale su
due file nascosti \texttt{.public} e \texttt{.private} e usate poi per criptare e decriptare i messaggi.
Ogni utente possiede una coppia di chiavi, quindi se un utente A deve spedire un messaggio
ad un utente B, quest'ultimo deve possedere la chiave pubblica di A per poter decifrare
il messaggio, una volta ricevuto.